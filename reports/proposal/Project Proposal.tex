\documentclass[11pt]{article}
\usepackage{hyperref}
\usepackage{listings}
\usepackage{graphicx}
\usepackage{float}
\usepackage{wrapfig}
\title{EECS 433 Proposal: Graph Databases}
\author{John Gunderman\\
		Matthew Varley\\
		Umang Banugaria}
\date{}
\begin{document}

\maketitle
\tableofcontents

\section{Problem Definition}
Implement an in-memory graph database. The uses for an in-memory graph database include testing frameworks and high-performance computations.

\section{Preliminary Goals of Project}
By the end of the term we aim to achieve the following:

\begin{itemize}
\item Implement a query-able graph datastore for general usage on the local machine.
\item Given enough time, a server component could be constructed to allow off-site queries of the database.
\item Implement a sensible in-memory storage solution, optimized for graph traversals. The storage API should be modular to allow for different back-ends to be implemented.
\item Given enough time, on-disk storage solutions could be considered
Implement a query API to allow searches based on node value and the relationships between nodes.
\end{itemize}

\section{Division of Work}

John - Project Lead, Queries \\
Matt - Storage \\
Umang -  Queries \\

\section{Papers}

\cite{Star} \cite{Cheng} \cite{Ma} \cite{Sun} \cite{Fan} \cite{He}

\bibliographystyle{IEEEtran}
\bibliography{bib}

\end{document}
